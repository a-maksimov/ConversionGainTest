%%% eskdx
\usepackage{eskdchngsheet}		% вставка листа регистрации изменений
\usepackage{eskdfreesize}		% вставка листов нестандартного формата

%%% математика
\usepackage{amsmath,amssymb}
\usepackage{gensymb}			% знак градуса командой \degree
\usepackage{icomma}				% запятая в десятичных дробях

%%% русский язык
\usepackage[T2A]{fontenc}		% поддержка русских букв
\usepackage[utf8]{inputenc}		% кодировка utf8
\usepackage[russian]{babel}		% языки: русский
\IfFileExists{pscyr.sty}{%
	\usepackage{pscyr}\renewcommand{\rmdefault}{ftm}}%
	{\usepackage{literat}}		% красивые русские шрифты
% решение проблемы копирования текста в буфер обмена
\input glyphtounicode.tex
\input glyphtounicode-cmr.tex	% from pdfx package
\pdfgentounicode=1
\usepackage{cmap}				% улучшенный поиск русских слов в полученном pdf-файле

%%% списки
\usepackage{enumitem}

%%% выравние на странице
\usepackage{varwidth}

%%% таблицы
\usepackage{multirow}
\usepackage{longtable,tabu}
\usepackage{float}				% опция [H] для точного размещения плавающих объектов
\usepackage{setspace}			% расстояния между строками в таблицах

%%% рисование
\usepackage[europeanresistors,americaninductors]{circuitikz} % рисование электрических схем (автоподключение tikz)
\usetikzlibrary{shapes,arrows.meta,chains} % библиотеки tikz

%%% базы данных
\usepackage{csvsimple}
%\newcommand{\csvloopx}[2][]{\csvloop{#1,#2}}
%\newcommand{\csvautotabularx}[2][]{\csvloopx[#1]{autotabular={#2}}}
%\newcommand{\respectpercent}{\catcode`\%=12\relax}

%%%
\usepackage{varwidth}

\usepackage{xr} 				% ссылки на внешние документы
\usepackage[ 					% гипертекстовое оглавление в PDF
	bookmarks=true, colorlinks=true, unicode=true,
	urlcolor=black,linkcolor=black, anchorcolor=black,
	citecolor=black, menucolor=black, filecolor=black,
]{hyperref}
\usepackage[russian]{cleveref}	% умные ссылки

%%% микротипографика
\usepackage[final]{microtype}[2016/05/14] % улучшает представление букв и слов в строках